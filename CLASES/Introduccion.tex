%###########################PRESENTACION##########################################
%Modo presentación
\documentclass[12pt]{beamer}

%Modo handout
%\documentclass[handout,compress]{beamer}
%\usepackage{pgfpages}
%\pgfpagesuselayout{4 on 1}[border shrink=1mm]

\usepackage{graphicx}
\usepackage{beamerthemeCambridgeUS}
\usepackage{subfig}
\usepackage{tikz}
\usepackage{amsmath}
\setbeamercovered{transparent}

\title[Introducción]{ANÁLISIS GEOESPACIAL}
\author[Edier Aristizábal]{Edier V. Aristizábal G.}
\institute{\emph{evaristizabalg@unal.edu.co}}
\date{(Versión:\today)}
\usepackage{textpos} 

\addtobeamertemplate{headline}{}{%
	\begin{textblock*}{2mm}(.9\textwidth,0cm)
	\hfill\includegraphics[height=1cm]{G:/Mi Unidad/ADMINISTRATIVA/logos/logo3.png}  
	\end{textblock*}
			}
%############################INICIO#############################################
\begin{document}
%###########################SLIDE
\begin{frame}
\titlepage
\centering
	\includegraphics[width=5cm]{G:/Mi Unidad/ADMINISTRATIVA/logos/unal.png}\hspace*{4.75cm}~%
   	\includegraphics[width=2cm]{G:/Mi Unidad/INVESTIGACION/GRUPOS DE INVESTIGACION/GEOHAZARDS/logo2} 
\end{frame}
%#############################SLIDE
\begin{frame}
\centering
	\includegraphics[width=12cm]{G:/Mi unidad/CATEDRA/ANALISIS GEOESPACIAL/fig/curso}
\end{frame}
%#############################SLIDE
 \begin{frame}
\centering
	\includegraphics[width=12cm]{G:/Mi unidad/CATEDRA/ANALISIS GEOESPACIAL/fig/github}
\end{frame}
 %#############################SLIDE
\begin{frame}
\frametitle{Objetivos del curso}
\framesubtitle{Objetivos y alcances del curso}
El curso \alert{Análisis Geoespacial} está orientado para estudiantes de posgrados que deseen adquirir conocimientos sobre sensores remotos y datos geoespaciales en un contexto ambiental, utilizando herramientas tipo Sistemas de Información Geográfica (SIG), Google Earth Engine (GEE), Big Data, y programación en lenguaje Python.\vfill

El curso es teórico - práctico. Se dictarán clases teóricas con las técnicas y modelos a utilizar, y clases prácticas donde se resolverán dudas con el manejo de las herramientas. El curso se evaluará a través de un trabajo individual durante todo el curso, donde el estudiante implementará en una cuenca de su elección las herramientas de análisis presentadas en el curso.
\end{frame}
 %#############################SLIDE
\begin{frame}
\centering
	\includegraphics[width=10cm]{G:/Mi unidad/CATEDRA/ANALISIS GEOESPACIAL/fig/ven}
\end{frame}
%#############################SLIDE
\begin{frame}
\centering
	\includegraphics[width=12cm]{G:/Mi unidad/CATEDRA/ANALISIS GEOESPACIAL/fig/progra}
\end{frame}
%#############################SLIDE
\begin{frame}
\centering
	\includegraphics[width=12cm]{G:/Mi unidad/CATEDRA/ANALISIS GEOESPACIAL/fig/program2}
\end{frame}
%#############################SLIDE
\begin{frame}
\centering
	\includegraphics[width=12cm]{G:/Mi unidad/CATEDRA/ANALISIS GEOESPACIAL/fig/eval}
\end{frame}
%#############################SLIDE
\end{document}